%Page setup
\documentclass[12pt, a4paper]{article}

%Setup margins
\usepackage{geometry}
\geometry{a4paper, portrait, margin=0.75in}

%Set Abstract title to a sensible size
\renewcommand{\abstractname}{\Large Abstract}

%Times for main font and math
\usepackage{mathptmx}

% Natbib references
\usepackage[numbers]{natbib}

%Colour
\usepackage[dvipsnames, table]{xcolor}
\newcommand{\shadeCell}
{
\cellcolor{black!25}
}

% Hyperref:
% This package makes all references within your document clickable. By default, these references will become boxed and colored. This is turned back to normal with the \hypersetup command below.
\usepackage[pdfpagelabels]{hyperref} %pdfpagelabels allows page numbers such as i and ii in pdf reader
\hypersetup{colorlinks=false,pdfborder=0 0 0}

\usepackage{etoolbox} %Let's let bibliography be ragged right to avoid typesetting problems
\apptocmd{\thebibliography}{\raggedright}{}{}

%\usepackage{pgfplotstable}
%\pgfplotsset{compat=1.12} %Compile the same on everything


%Appendix Handling
\usepackage[titletoc,title]{appendix}

%Acronyms!
\usepackage[nonumberlist, acronyms, toc,]{glossaries}
\setacronymstyle{long-short}
\renewcommand{\glsgroupskip}{} %Don't put a small space between groups
\newacronym{fpga}{FPGA}{field programmable gate array}
\newacronym{ram}{RAM}{random access memory}
\newacronym{rom}{ROM}{read only memory}
\newacronym{led}{LED}{light emitting diode}
\newacronym{alu}{ALU}{arithmetic logic unit}
\newacronym{lsb}{LSB}{least significant bit}
\newacronym{msb}{MSB}{most significant bit}
\newacronym{bcd}{BCD}{binary coded decimal} %Import list of acronyms

%Mathematical symbols
\usepackage{amssymb}

%Sensible Re and Im
\usepackage{physics}

%si representations
\usepackage{siunitx}

%Drawing
\usepackage{tikz}
\usetikzlibrary{shapes.geometric, shapes.misc, arrows}
\usetikzlibrary{calc}
\usetikzlibrary{positioning}
\usetikzlibrary{patterns}
\usetikzlibrary{automata}

\newcommand{\tikzNodeDist}{1.5cm}
\newcommand{\tikzStateNodeDist}{3cm}

\tikzset{font={\small}} %set small font
\tikzstyle{every state}=[fill=black!40,draw=none,minimum size=1.75cm] %Setup state machine style
\tikzstyle{stateMachine}=[->,auto,node distance=\tikzStateNodeDist,thick]

\tikzstyle{normalBlock} = [rectangle, rounded corners, minimum width=\tikzNodeDist, minimum height=\tikzNodeDist, text centered, draw=black, fill=black!40]
\tikzstyle{arrowNml} = [thick,->,>=stealth]
\tikzstyle{arrowRev} = [thick,<-,>=stealth]
\tikzstyle{arrowDbl} = [thick,<->,>=stealth]
\tikzstyle{lineNml} = [thick]
\newcommand{\buswidth}[5]{\draw ([xshift=#4,yshift=#5] #1) node [#3=.6ex,] {#2} +(45:-.8ex) -- +(45:.8ex)}
\newcommand{\buswidthoffset}{-0.15*\tikzNodeDist}

%Pretty plots
\usepackage{pgfplots}
\pgfplotsset{compat=newest}
% We have some pretty big data sets, so lets use the externalise feature
% See http://tex.stackexchange.com/questions/7953/how-to-expand-texs-main-memory-size-pgfplots-memory-overload
%\usepgfplotslibrary{external} 
%\tikzexternalize

% Pretty tables
\usepackage{tabularx, booktabs}
\renewcommand{\tabularxcolumn}[1]{>{\small}m{#1}} %Vertically centre with X
\newcolumntype{Y}{>{\centering\arraybackslash}X}
%\newcolumntype{Y}{}

\usepackage{multirow}

% Caption. For better looking captions the captions.
\usepackage{caption}
\usepackage{subcaption}

\usepackage{MnSymbol} % Right arrow and other symbols

%Listings
\usepackage{listings,chngcntr}
% set the default code style
\lstset{
	frame=tb, % draw a frame at the top and bottom of the code block
	tabsize=4, % tab space width
	breaklines=true, %Wrap text
	showstringspaces=false, % don't mark spaces in strings
	numbers=left, % display line numbers on the left
	commentstyle=\color{ForestGreen}, % comment color
	keywordstyle=\color{blue}, % keyword color
	stringstyle=\color{red}, % string color
	basicstyle=\small, %normal font size
	postbreak=\raisebox{0ex}[0ex][0ex]{\ensuremath{\rcurvearrowse\space}} %Chaacter to display on broken line
}


%Make heading say "list of listings", not just "listings"
\renewcommand\lstlistlistingname{List of Listings}

%% Allow listing line break at any number
%% http://tex.stackexchange.com/questions/65743/listings-package-does-not-break
%\makeatletter
%\def\lst@lettertrue{\let\lst@ifletter\iffalse}
%\makeatother

%Allow easy typesetting of SI units
\usepackage{siunitx}

%Allow ragged right tables
\usepackage{array}

\usepackage{textcomp}

%Allow floatbarrier
\usepackage{placeins}

%Timing diagrams
\usepackage{tikz-timing}[2011/01/09]

%Posh enumeration

\usepackage{enumitem}

\setlist[enumerate]{nosep} %No special separation for enumerated lists

\setlist[itemize]{nosep}

%Review command for red text
\newcommand{\review}[1]
{
	{\color{red} \bfseries #1}
}