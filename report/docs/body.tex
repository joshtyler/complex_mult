\section{Introduction} \label{sec:intro}

\section{Design} \label{sec:design}
\review{Include `Circuit diagrams' and discussion of design}

\begin{figure}[ht]
	\centering
	
	\begin{tikzpicture}[node distance=\tikzNodeDist]
	
	%Blocks
	\node (dbnce) [normalBlock] at (0,0) {\shortstack{Debouncers}};
	\node (sm) [normalBlock, right= of dbnce.south east, anchor=south west, minimum height = 2*\tikzNodeDist] {\shortstack{State\\Machine}};
	\node (mult) [normalBlock, below= of sm] {\shortstack{Multiplier}};
	\node (dispsw) [normalBlock, above= of sm] {\shortstack{Switch\\Display\\Decoder}};
	\node (displed) [normalBlock, right= of sm] {\shortstack{LED\\Display\\Decoder}};
	
	%Input side arrows
	\draw[arrowRev] (dbnce.west) -- ++(-0.5*\tikzNodeDist,0) coordinate(swcorner) to node[left] {Switches[9:8]} ++(0,\tikzNodeDist) coordinate(swfork) -- ++(-\tikzNodeDist,0) coordinate(swin) node[left] {Switches[9:0]};
	
	\draw[arrowNml] (swfork) to node[above] {Switches[7:0]} (sm.west |- swfork.north);
	
	\draw[arrowNml] ([xshift=-0.5*\tikzNodeDist] sm.west |- swfork.north) |- (dispsw);
	
	\draw[arrowNml] (dbnce) -- (sm.west |- dbnce);
	
	\buswidth{swfork}{$8$}{above}{-\buswidthoffset}{0};
	\buswidth{swin}{$10$}{above}{-\buswidthoffset}{0};
	\buswidth{swcorner}{$2$}{above}{-\buswidthoffset}{0};
	\buswidth{dbnce.east}{$2$}{above}{-\buswidthoffset}{0};
	
	%Output side arrows
	\draw[arrowNml] (displed.east) -- ++(\tikzNodeDist,0) coordinate(output) node[right] {\shortstack{LED\\7 Segment\\Displays}};
	\buswidth{displed.east}{$21$}{above}{-\buswidthoffset}{0};
	
	\draw[arrowNml] (dispsw.east) -- (output |- dispsw.east) node[right] {\shortstack{Switch\\7 Segment\\Displays}};
	\buswidth{dispsw.east}{$21$}{above}{-\buswidthoffset}{0};
	
	
	\draw[arrowNml] (sm) to node[above] {LEDs} (displed);
	\buswidth{sm.east}{$8$}{below}{-\buswidthoffset}{0};
	
	\draw[arrowNml] ([xshift=0.5*\tikzNodeDist] sm.east) |- ([yshift=-\tikzNodeDist]output) node[right] {LEDs};
	
	%Multiplier arrows
	\coordinate(multstart1) at ([xshift=0.25*\tikzNodeDist] sm.south west);
	\coordinate(multend1) at ( multstart1 |- mult.north);
	\coordinate(multend2) at ([xshift=-0.25*\tikzNodeDist] sm.south east);
	\coordinate(multstart2) at ( multend2 |- mult.north);
	
	\draw[arrowNml] (multstart1) to node[left] {\shortstack{$4 \times$ 8-bit\\Input Words}} (multend1);
	\draw[arrowNml] (multstart2) to node[right] {\shortstack{$2 \times$ 8-bit\\Output Words}} ( multend2);
	
	\buswidth{multstart1}{$32$}{right}{0}{\buswidthoffset};
	\buswidth{multstart2}{$16$}{left}{0}{-\buswidthoffset};
	
	\end{tikzpicture}
	\caption{Overall Block Diagram}
	\label{fig:overall-block-diag}
\end{figure}

\subsection{Complex Multiplier} \label{sec:design-mult}

\begin{figure}[ht]
	\centering
	
	\begin{tikzpicture}[node distance=\tikzNodeDist]
	
	%Blocks
	\node (mult1) [normalBlock] at (0,0) {\shortstack{Multiplier\\1}};
	\coordinate (middlem1m2) at ([xshift=0.5*\tikzNodeDist] mult1.south east);
	\node (mult2) [normalBlock, right= of mult1] {\shortstack{Multiplier\\2}};
	\node (mult3) [normalBlock, right= of mult2] {\shortstack{Multiplier\\3}};
	\coordinate (middlem3m4) at ([xshift=0.5*\tikzNodeDist] mult3.south east);
	\node (mult4) [normalBlock, right= of mult3] {\shortstack{Multiplier\\4}};
	
	\node (sub) [normalBlock, below= of middlem1m2] {\shortstack{Subtractor}};
	\node (add) [normalBlock, below= of middlem3m4] {\shortstack{Adder}};
	
	%Inputs
	\draw[arrowRev] ([xshift=-0.25*\tikzNodeDist] mult1.north) -- ++(0,0.5*\tikzNodeDist) coordinate(in1) node[above] {$a$};
	\draw[arrowRev] ([xshift=0.25*\tikzNodeDist] mult1.north) -- ++(0,0.5*\tikzNodeDist) coordinate(in2) node[above] {$c$};
	
	\draw[arrowRev] ([xshift=-0.25*\tikzNodeDist] mult2.north) -- ++(0,0.5*\tikzNodeDist) coordinate(in3) node[above] {$b$};
	\draw[arrowRev] ([xshift=0.25*\tikzNodeDist] mult2.north) -- ++(0,0.5*\tikzNodeDist) coordinate(in4) node[above] {$d$};
	
	\draw[arrowRev] ([xshift=-0.25*\tikzNodeDist] mult3.north) -- ++(0,0.5*\tikzNodeDist) coordinate(in5) node[above] {$a$};
	\draw[arrowRev] ([xshift=0.25*\tikzNodeDist] mult3.north) -- ++(0,0.5*\tikzNodeDist) coordinate(in6) node[above] {$d$};
	
	\draw[arrowRev] ([xshift=-0.25*\tikzNodeDist] mult4.north) -- ++(0,0.5*\tikzNodeDist) coordinate(in7) node[above] {$b$};
	\draw[arrowRev] ([xshift=0.25*\tikzNodeDist] mult4.north) -- ++(0,0.5*\tikzNodeDist) coordinate(in8) node[above] {$c$};
	
	\buswidth{in1}{$15$}{left}{0}{\buswidthoffset};
	\buswidth{in2}{$15$}{right}{0}{\buswidthoffset};
	\buswidth{in3}{$15$}{left}{0}{\buswidthoffset};
	\buswidth{in4}{$15$}{right}{0}{\buswidthoffset};
	\buswidth{in5}{$15$}{left}{0}{\buswidthoffset};
	\buswidth{in6}{$15$}{right}{0}{\buswidthoffset};
	\buswidth{in7}{$15$}{left}{0}{\buswidthoffset};
	\buswidth{in8}{$15$}{right}{0}{\buswidthoffset};
	
	%Intermediate signals
	\draw[arrowNml] (mult1) |- (sub);
	\draw[arrowNml] (mult2) |- (sub);
	\draw[arrowNml] (mult3) |- (add);
	\draw[arrowNml] (mult4) |- (add);
	
	\buswidth{mult1.south}{$30$}{left}{0}{-0.15*\tikzNodeDist};
	\buswidth{mult2.south}{$30$}{left}{0}{-0.15*\tikzNodeDist};
	\buswidth{mult3.south}{$30$}{left}{0}{-0.15*\tikzNodeDist};
	\buswidth{mult4.south}{$30$}{left}{0}{-0.15*\tikzNodeDist};
	
	% Outputs
	\draw[arrowNml] (sub.south) -- ++(0,-0.5*\tikzNodeDist) node[below] {$\Re(\text{result})$};
	\draw[arrowNml] (add.south) -- ++(0,-0.5*\tikzNodeDist) node[below] {$\Im(\text{result})$};
	
	\buswidth{sub.south}{$30$}{left}{0}{\buswidthoffset};
	\buswidth{add.south}{$30$}{left}{0}{\buswidthoffset};
	
	\end{tikzpicture}
	\caption{Multiplier architecture}
	\label{fig:mult-arch}
\end{figure}

\subsection{State Machine} \label{sec:design-sm}

\begin{figure}[ht]
	\centering
	\begin{tikzpicture}[stateMachine]
	
	\node[initial,state] (halt) {\texttt{HALT}};
	\node[state]         (wait1) [right of=halt] {\texttt{WAIT\_1}};
	\node[state]         (wait0) [right of=wait1] {\texttt{WAIT\_0}};
	\node[state]         (done) [right of=wait0] {\texttt{DONE}};
	
	
	\path (halt) edge[bend left] node[auto] {\texttt{run}} (wait1)
	(wait1) edge[bend left] node[auto] {\texttt{handshake}} (wait0)
	(wait0) edge[bend left] node[auto] {\texttt{!handshake}} (done)
	(done) edge[bend left] node[auto] {} (halt);
	
	\end{tikzpicture}
	\caption{Read state machine state transition diagram}
	\label{fig:read-sm-transition}
\end{figure}

\begin{figure}[ht]
	\centering
	\begin{tikzpicture}[stateMachine]
	
	\node[initial,state] (realA) {\texttt{REAL\_A}};
	\node[state]         (imagA) [right of=realA] {\texttt{IMAG\_A}};
	\node[state]         (realQ) [right of=imagA] {\texttt{REAL\_Q}};
	\node[state]         (imagQ) [right of=realQ] {\texttt{IMAG\_Q}};
	\coordinate (middle) at ([xshift=0.5*\tikzStateNodeDist] imagA);
	
	\node[state]         (dispR) [below of=middle] {\texttt{\shortstack{DISP\_\\REAL}}};
	\node[state]         (dispI) [below of=dispR] {\texttt{\shortstack{DISP\_\\IMAG}}};
	
	\path (realA) edge[bend left] node[auto] {\texttt{read\_done}} (imagA)
	(imagA) edge[bend left] node[auto] {\texttt{read\_done}} (realQ)
	(realQ) edge[bend left] node[auto] {\texttt{read\_done}} (imagQ)
	(imagQ) edge[bend left] node[auto] {\texttt{read\_done}} (dispR)
	(dispR) edge[] node[auto] {\texttt{handshake}} (dispI)
	(dispI) edge[bend left] node[auto] {\texttt{!handshake}} (realA);
	
	\end{tikzpicture}
	\caption{Main state machine state transition diagram}
	\label{fig:main-sm-transition}
\end{figure}

\section{Verification} \label{sec:verif}
\review{Include sample data and results}
\review{Add asterisked out states manually}
\review{check waveforms}
\review{Fix line numbers for stimulus}

%%%%%%%%%%%%%

\lstinputlisting[language=verilog, caption={\texttt{test\_bin\_to\_bcd.sv} Stimulus}, label={lst:test-bin-to-bcd},firstnumber=17,linerange={17-37}]{../hdl/sim/test_bin_to_bcd.sv}
\begin{figure}[ht]
	\centering
	\begin{tikztimingtable} [xscale=2.0]
	in[7:0] & 1D{0x80} 1D{0x81} \\
	sign & H H \\
	hundreds[3:0] & 2D{0x1} \\
	tens[3:0] & 2D{0x2} \\
	units[3:0] & 1D{0x8} 1D{0x7} \\
	disp[3][6:0] & 2D{0x40} \\
	disp[2][6:0] & 2D{0x6} \\
	disp[1][6:0] & 2D{0x5b} \\
	disp[0][6:0] & 1D{0x7f} 1D{0x7} \\
\end{tikztimingtable}

	\caption{\texttt{test\_bin\_to\_bcd.sv} Output}
	\label{fig:test-bin-to-bcd}
\end{figure}

%%%%%%%%%%%%%

%%%%%%%%%%%%%

\lstinputlisting[language=verilog, caption={\texttt{test\_cmplx\_mult.sv} Stimulus}, label={lst:test-cmplx-mult},firstnumber=17,linerange={17-37}]{../hdl/sim/test_cmplx_mult.sv}
\begin{figure}[ht]
	\centering
	%Verified
\begin{tikztimingtable} [xscale=3]
	clk & 13{U} \\
	reset\_n & H L H H H H H H H H H H H \\
	handshake & L L L H L H L H L H L H L \\
	data\_in[7:0] & 3D{0x0} 2D{0xd0} 2D{0xe7} 2D{0x21} 4D{0xe9} \\
	LED[7:0] & 10.5D{0x0} D{0xde} D{0x4} 0.5D{} \\
	HEX0[6:0] & 3D{0x40} 2D{0x0} 2D{0x12} 6D{0x30} \\
	HEX1[6:0] & 3D{0x40} 2D{0x19} 2D{0x24} 2D{0x30} 4D{0x24} \\
	HEX2[6:0] & 13D{0x40} \\
	HEX3[6:0] & 3D{0x7f} 4D{0x3f} 2D{0x7f} 4D{0x3f} \\
	HEX4[6:0] & 10.5D{0x40} 2D{0x19} 0.5D{} \\
	HEX5[6:0] & 10.5D{0x40} D{0x30} 1.5D{0x40} \\
	HEX6[6:0] & 13D{0x40} \\
	HEX7[6:0] & 10.5D{0x7f} D{0x3f} 1.5D{0x7f} \\
\end{tikztimingtable}

	\caption{\texttt{test\_cmplx\_mult.sv} Output}
	\label{fig:test-cmplx-mult}
\end{figure}

%%%%%%%%%%%%%

%%%%%%%%%%%%%

\lstinputlisting[language=verilog, caption={\texttt{test\_debounce.sv} Stimulus}, label={lst:test-debounce},firstnumber=17,linerange={17-37}]{../hdl/sim/test_debounce.sv}
\begin{figure}[ht]
	\centering
	%Verified
\begin{tikztimingtable} [xscale=0.75]
	clk & 59{C} \\
	signal\_in & L L L L L L L L L L L L L L L L L L L L L L H H H H L L H H H H H H H H H H H H H H H H H H H H H H H H H H H H H H H \\
	signal\_out & X X X X X X X X X X X X X X X X X X X X X L L L L L L L L L L L L L L L L L L L L L L L L L L L L L L H H H H H H H H \\
	\extracode
	\begin{pgfonlayer}{background}
		\vertlines[gray!40]{1,3,...,57}
	\end{pgfonlayer}
\end{tikztimingtable}

	\caption{\texttt{test\_debounce.sv} Output}
	\label{fig:test-debounce}
\end{figure}

%%%%%%%%%%%%%

%%%%%%%%%%%%%

\lstinputlisting[language=verilog, caption={\texttt{test\_mult.sv} Stimulus}, label={lst:test-mult},firstnumber=17,linerange={17-37}]{../hdl/sim/test_mult.sv}
\begin{figure}[ht]
	\centering
	%Verified
\begin{tikztimingtable} [xscale=5.0]
	re\_x[14:0] & 1D{0x751f} 1D{0x7508} 1D{0x6bf9} 1D{0x27de} 1D{0x2dc1} 1D{0x27cd} 1D{0x46e5} 1D{0x31e0} 1D{0x7052} \\
	im\_x[14:0] & 1D{0x7702} 1D{0x7c42} 1D{0x6a6d} 1D{0x925} 1D{0x41fb} 1D{0x2bd3} 1D{0xbd2} 1D{0x512e} 1D{0x61ae} \\
	re\_y[14:0] & 1D{0x285a} 1D{0x4cdf} 1D{0x4ace} 1D{0x5960} 1D{0x3e3e} 1D{0x34bf} 1D{0x18de} 1D{0x7341} 1D{0x43f9} \\
	im\_y[14:0] & 1D{0x7878} 1D{0x2453} 1D{0x75a1} 1D{0x6f1a} 1D{0x6469} 1D{0x827} 1D{0x52ce} 1D{0x8d2} 1D{0x4b92} \\
	re\_z[29:0] & 1D{0x3e054bf6} 1D{0x2b8c092} 1D{0x3499cd1} 1D{0x3a96a87e} 1D{0x470b6cb} 1D{0x6ce10ce} 1D{0x3c8a2b9a} 1D{0x3f214224} 1D{0x3d778686} \\
	im\_z[29:0] & 1D{0x3ee7153c} 1D{0x3f30e716} 1D{0x54b5a4f} 1D{0x3bfd196c} 1D{0x2bfd6cf3} 1D{0xa4c06a8} 1D{0xb3ad662} 1D{0x40cb06e} 1D{0xa522102} \\
\end{tikztimingtable}

	\caption{\texttt{test\_mult.sv} Output}
	\label{fig:test-mult}
\end{figure}

%%%%%%%%%%%%%

%%%%%%%%%%%%%

\lstinputlisting[language=verilog, caption={\texttt{test\_read\_sm.sv} Stimulus}, label={lst:test-read-sm},firstnumber=17,linerange={17-37}]{../hdl/sim/test_read_sm.sv}
\begin{figure}[ht]
	\centering
	\begin{tikztimingtable} [xscale=2.0]
	clk & L L L H L H L H L H L H L H L H L H L H L H L H L H L H L H L H \\
	reset\_n & H L H H H H H H H H H H H H H H H H H H H H H H H H H H H H H H \\
	handshake & L L L L L L L L L L L L L H H H H H H H H H H L L L L L L L L L \\
	state[3:0] & 7D{HALT} 16D{****} 2D{DONE} 7D{HALT} \\
	run & L L L L L H H L L L L L L L L L L L L L L L L L L L L L L L L L \\
	read & X L L L L L L L L L L L L H H L L L L L L L L L L L L L L L L L \\
	done & L L L L L L L L L L L L L L L L L L L L L L L H H L L L L L L L \\
\end{tikztimingtable}

	\caption{\texttt{test\_read\_sm.sv} Output}
	\label{fig:test-read-sm}
\end{figure}

%%%%%%%%%%%%%

%%%%%%%%%%%%%

\lstinputlisting[language=verilog, caption={\texttt{test\_sm.sv} Stimulus}, label={lst:test-sm},firstnumber=17,linerange={17-37}]{../hdl/sim/test_sm.sv}
\begin{figure}[ht]
	\centering
	\begin{tikztimingtable} [xscale=2.0]
%	clk & L L L H L H L H L H L H L H L H L H L H L H L H L H L H L H L H L H L H L H L H L H L H L H L H L H L H L H L H L H L H L H L H L H L H L H L H L H L H L H L H L H L H L H L H L H L H L H L H L H L H L H L H L H L H L H L H L H L H L H L H L H L H L H L H L H L H L H L H L H L H L H L H L H L H L H L H L H L H L H L H L H L H L H L H L H L H L H L H L H L H L H L H L H L H L H L H L H L H L H L H L H L H L L H L H L H L H L H L H L H L H L H L H L H L H L H L H L H L H L H L H L H L H L H L H L H L H L H L H L H L H L H L H L H L H L H L H L H L H L H L H L H L H L H L H L H L H L H L H L H L H L H L H L H L H L H L H L H L H L H L H L H L H L H L H L H L H L H L H L H L H L H L H L H L H L H L H L H L H L H L H L H L H L H L H L H L H L H L H L H L H L H L H L H L H L H L H L H L H L H L H L H L H L L H L H L H L H L H L H L H L H L H L H L H L H L H L H L H L H L H L H L H L H L H L H L H L H L H L H L H L H L H L H L H L H L H L H L H L H L H L H L H L H L H L H L H L H L H L H L H L H L H L H L H L H L H L H L H L H L H L H L H L H L H L H L H L H L H L H L H L H L H L H L H L H L H L H L H L H L H L H L H L H L H L H L H L H L H L H L H L H L H L H L H L H L H L H L H L H L H L H L H L H L H L H L H L H L H L H L H L H L H L H L H L H L H L H L H L H L H L H L H L H L H L H L H L H L H L H L H L H L H L H L H L H L H L H L H L H L H L H L H L H L H L H L H L H L H L H L H L H L H L H L H L H L H L H L H L H L H L H L H L H L H L H L H L H L H L H L H L H L H L H L H L H L H L H L H L H L H L H L H L H L H L H L H L H L H L H L H L H L H L H L H L H L H L H L H L H L H L H L H L H L L H L H L H L H L H L H L H L H L H L H L H L H L H L H L H L H L H L H L H L H L H L H L H L H L H L H L H L H L H L H L H L H L H L H L H L H L H L H L H L H L H L H L H L H L H L H L H L H L H L H L H L H L H L H L H L H L H L H L H L H L H L H L H L H L H L H L H L H L H L H L H L H L H L H L H L H L H L H L H L H L H L H L H L H L H L H L H L H L H L H L H L H L H L H L H L H L H L H L H L H L L H L H L H L H L H L H L H L H L H L H L H L H L H L H L H L H L H L H L H L H L H L H L H L H L H L H L H L H L H L H L H L H L H L H L H L H L H L H L H L H L H L H L H L H L H L H L H L H L H L H L H L H L H L H L H L H L H L H L H L H L H L H L H L H L H L H L H L H L H L H L H L H L H L H L H L H L H L H L H L H L H L H L H L H L H L H L H L H L H L H L H L H L H L H L H L H L H L H L H L H L H L H L H L H L H L H L H L H L H L H L H L H L H L H L H L H L H L H L H L H L H L H L H L H L H L H L H L H L H L H L H L H L H L H L H L H L H L H L H L H L H L H L H L H L H L H L H L H L H L H L H L H L H L H L H L H L H L H L H L H L H L H L H L H L H L H L H L H L H L H L H L H L H L H L H L H L H L H L H L H L H L H L H L H L H L H L H L H L H L H L H L H L H L H L H L H L H L H L H L H L L H L H L H L H L H L H L H L H L H L H L H L H L H L H L H L H L H L H L H L H L H L H L H L H L H L H L H L H L H L H L H L H L H L H L H L H L H L H L H L H L H L H L H L H L H L H L H L H L H L H L H L H L H L H L H L H L H L H L H L H L H L H L H L H L H L H L H L H L H L H L H L H L H L H L H L H L H L H L H L H L H L H L H L H L H L H L H L H L H L H L H L H L H L H L H L H L H L H L H L H L L H L H L H L H L H L H L H L H L H L H L H L H L H L H L H L H L H L H L H L H L H L H L H L H L H L H L H L H L H L H L H L H L H L H L H L H L H L H L H L H L H L H L H L H L H L H L H L H L H L H L H L H L H L H L H L H L H L H L H L H L H L H L H L H L H L H L H L H L H L H L H L H L H L H L H L H L H L H L H L H L H L H L H L H L H L H L H L H L H L H L H L H L H L H L H L H L H L H L H L H L H L H L H L H L H L H L H L H L H L H L H L H L H L H L H L H L H L H L H L H L H L H L H L H L H L H L H L H L H L H L H L H L H L H L H L H L H L H L H L H L H L H L H L H L H L H L H L H L H L H L H L H L H L H L H L H L H L H L H L H L H L H L H L H L H L H L H L H L H L H L H L H L H L H L H L H L H L H L H L H L H L H L H L H L H L H L H L H L H L H L H L H L H L H L H L H L H L H L H L H L L H L H L H L H L H L H L H L H L H L H L H L H L H L H L H L H L H L H L H L H L H L H L H L H L H L H L H L H L H L H L H L H L H L H L H L H L H L H L H L H L H L H L H L H L H L H L H L H L H L H L H L H L H L H L H L H L H L H L H L H L H L H L H L H L H L H L H L H L H L H L H L H L H L H L H L H L H L H L H L H L H L H L H L H L H L H L H L H L H L H L H L H L H L H L H L H L H L H L H L H L L H L H L H L H L H L H L H L H L H L H L H L H L H L H L H L H L H L H L H L H L H L H L H L H L H L H L H L H L H L H L H L H L H L H L H L H L H L H L H L H L H L H L H L H L H L H L H L H L H L H L H L H L H L H L H L H L H L H L H L H L H L H L H L H L H L H L H L H L H L H L H L H L H L H L H L H L H L H L H L H L H L H L H L H L H L H L H L H L H L H L H L H L H L H L H L H L H L H L H L H L L H L H L H L H L H L H L H L H L H L H L H L H L H L H L H L H L H L H L H L H L H L H L H L H L H L H L H L H L H L H L H L H L H L H L H L H L H L H L H L H L H L H L H L H L H L H L H L H L H L H L H L H L H L H L H L H L H L H L H L H L H L H L H L H L H L H L H L H L H L H L H L H L H L H L H L H L H L H L H L H L H L H L H L H L H L H L H L H L H L H L H L H L H L H L H L H L H L H L H L H L L H L H L H L H L H L H L H L H L H L H L H L H L H L H L H L H L H L H L H L H L H L H L H L H L H L H L H L H L H L H L H L H L H L H L H L H L H L H L H L H L H L H L H L H L H L H L H L H L H L H L H L H L H L H L H L H L H L H L H L H L H L H L H L H L H L H L H L H L H L H L H L H L H L H L H L H L H L H L H L H L H L H L H L H L H L H L H L H L H L H L H L H L H L H L H L H L H L H L H L H L L H L H L H L H L H L H L H L H L H L H L H L H L H L H L H L H L H L H L H L H L H L H L H L H L H L H L H L H L H L H L H L H L H L H L H L H L H L H L H L H L H L H L H L H L H L H L H L H L H L H L H L H L H L H L H L H L H L H L H L H L H L H L H L H L H L H L H L H L H L H L H L H L H L H L H L H L H L H L H L H L H L H L H L H L H L H L H L H L H L H L H L H L H L H L H L H L H L H L H L H \\
%	reset\_n & H L H H H H H H H H H H H H H H H H H H H H H H H H H H H H H H H H H H H H H H H H H H H H H H H H H H H H H H H H H H H H H H H H H H H H H H H H H H H H H H H H H H H H H H H H H H H H H H H H H H H H H H H H H H H H H H H H H H H H H H H H H H H H H H H H H H H H H H H H H H H H H H H H H H H H H H H H H H H H H H H H H H H H H H H H H H H H H H H H H H H H H H H H H H H H H H H H H H H H H H H H H H H H H H H H H H H H H H H H H H H H H H H H H H H H H H H H H H H H H H H H H H H H H H H H H H H H H H H H H H H H H H H H H H H H H H H H H H H H H H H H H H H H H H H H H H H H H H H H H H H H H H H H H H H H H H H H H H H H H H H H H H H H H H H H H H H H H H H H H H H H H H H H H H H H H H H H H H H H H H H H H H H H H H H H H H H H H H H H H H H H H H H H H H H H H H H H H H H H H H H H H H H H H H H H H H H H H H H H H H H H H H H H H H H H H H H H H H H H H H H H H H H H H H H H H H H H H H H H H H H H H H H H H H H H H H H H H H H H H H H H H H H H H H H H H H H H H H H H H H H H H H H H H H H H H H H H H H H H H H H H H H H H H H H H H H H H H H H H H H H H H H H H H H H H H H H H H H H H H H H H H H H H H H H H H H H H H H H H H H H H H H H H H H H H H H H H H H H H H H H H H H H H H H H H H H H H H H H H H H H H H H H H H H H H H H H H H H H H H H H H H H H H H H H H H H H H H H H H H H H H H H H H H H H H H H H H H H H H H H H H H H H H H H H H H H H H H H H H H H H H H H H H H H H H H H H H H H H H H H H H H H H H H H H H H H H H H H H H H H H H H H H H H H H H H H H H H H H H H H H H H H H H H H H H H H H H H H H H H H H H H H H H H H H H H H H H H H H H H H H H H H H H H H H H H H H H H H H H H H H H H H H H H H H H H H H H H H H H H H H H H H H H H H H H H H H H H H H H H H H H H H H H H H H H H H H H H H H H H H H H H H H H H H H H H H H H H H H H H H H H H H H H H H H H H H H H H H H H H H H H H H H H H H H H H H H H H H H H H H H H H H H H H H H H H H H H H H H H H H H H H H H H H H H H H H H H H H H H H H H H H H H H H H H H H H H H H H H H H H H H H H H H H H H H H H H H H H H H H H H H H H H H H H H H H H H H H H H H H H H H H H H H H H H H H H H H H H H H H H H H H H H H H H H H H H H H H H H H H H H H H H H H H H H H H H H H H H H H H H H H H H H H H H H H H H H H H H H H H H H H H H H H H H H H H H H H H H H H H H H H H H H H H H H H H H H H H H H H H H H H H H H H H H H H H H H H H H H H H H H H H H H H H H H H H H H H H H H H H H H H H H H H H H H H H H H H H H H H H H H H H H H H H H H H H H H H H H H H H H H H H H H H H H H H H H H H H H H H H H H H H H H H H H H H H H H H H H H H H H H H H H H H H H H H H H H H H H H H H H H H H H H H H H H H H H H H H H H H H H H H H H H H H H H H H H H H H H H H H H H H H H H H H H H H H H H H H H H H H H H H H H H H H H H H H H H H H H H H H H H H H H H H H H H H H H H H H H H H H H H H H H H H H H H H H H H H H H H H H H H H H H H H H H H H H H H H H H H H H H H H H H H H H H H H H H H H H H H H H H H H H H H H H H H H H H H H H H H H H H H H H H H H H H H H H H H H H H H H H H H H H H H H H H H H H H H H H H H H H H H H H H H H H H H H H H H H H H H H H H H H H H H H H H H H H H H H H H H H H H H H H H H H H H H H H H H H H H H H H H H H H H H H H H H H H H H H H H H H H H H H H H H H H H H H H H H H H H H H H H H H H H H H H H H H H H H H H H H H H H H H H H H H H H H H H H H H H H H H H H H H H H H H H H H H H H H H H H H H H H H H H H H H H H H H H H H H H H H H H H H H H H H H H H H H H H H H H H H H H H H H H H H H H H H H H H H H H H H H H H H H H H H H H H H H H H H H H H H H H H H H H H H H H H H H H H H H H H H H H H H H H H H H H H H H H H H H H H H H H H H H H H H H H H H H H H H H H H H H H H H H H H H H H H H H H H H H H H H H H H H H H H H H H H H H H H H H H H H H H H H H H H H H H H H H H H H H H H H H H H H H H H H H H H H H H H H H H H H H H H H H H H H H H H H H H H H H H H H H H H H H H H H H H H H H H H H H H H H H H H H H H H H H H H H H H H H H H H H H H H H H H H H H H H H H H H H H H H H H H H H H H H H H H H H H H H H H H H H H H H H H H H H H H H H H H H H H H H H H H H H H H H H H H H H H H H H H H H H H H H H H H H H H H H H H H H H H H H H H H H H H H H H H H H H H H H H H H H H H H H H H H H H H H H H H H H H H H H H H H H H H H H H H H H H H H H H H H H H H H H H H H H H H H H H H H H H H H H H H H H H H H H H H H H H H H H H H H H H H H H H H H H H H H H H H H H H H H H H H H H H H H H H H H H H H H H H H H H H H H H H H H H H H H H H H H H H H H H H H H H H H H H H H H H H H H H H H H H H H H H H H H H H H H H H H H H H H H H H H H H H H H H H H H H H H H H H H H H H H H H H H H H H H H H H H H H H H H H H H H H H H H H H H H H H H H H H H H H H H H H H H H H H H H H H H H H H H H H H H H H H H H H H H H H H H H H H H H H H H H H H H H H H H H H H H H H H H H H H H H H H H H H H H H H H H H H H H H H H H H H H H H H H H H H H H H H H H H H H H H H H H H H H H H H H H H H H H H H H H H H H H H H H H H H H H H H H H H H H H H H H H H H H H H H H H H H H H H H H H H H H H H H H H H H H H H H H H H H H H H H H H H H H H H H H H H H H H H H H H H H H H H H H H H H H H H H H H H H H H H H H H H H H H H H H H H H H H H H H H H H H H H H H H H H H H H H H H H H H H H H H H H H H H H H H H H H H H H H H H H H H H H H H H H H H H H H H H H H H H H H H H H H H H H H H H H H H H H H H H H H H H H H H H H H H H H H H H H H H H H H H H H H H H H H H H H H H H H H H H H H H H H H H H H H H H H H H H H H H H H H H H H H H H H H H H H H H H H H H H H H H H H H H H H H H H H H H H H H H H H H H H H H H H H H H H H H H H H H H H H H H H H H H H H H H H H H H H H H H H H H H H H H H H H H H H H H H H H H H H H H H H H H H H H H H H H H H H H H H H H H H H H H H H H H H H H H H H H H H H H H H H H H H H H H H H H H H H H H H H H H H H H H H H H H H H H H H H H H H H H H H H H H H H H H H H H H H H H H H H H H H H H H H H H H H H H H H H H H H H H H H H H H H H H H H H H H H H H H H H H H H H H H H H H H H H H H H H H H H H H H H H H H H H H H H H H H H H H H H H H H H H H H H H H H H H H H H H H H H H H H H H H H H H H H H H H H H H H \\
%	handshake & L L L L L L L L L L L L L L L L L L L L L L L L L L L L L L L L L L L L L L L L L L L L L L L L L L L L L L L L L L L L L L L L L L L L L L L L L L L L L L L L L L L L L L L L L L L L L L L L L L L L L L L L L L L L L L L L L L L L L L L L L L L L L L L L L L L L L L L L L L L L L L L L L L L L L L L L L L L L L L L L L L L L L L L L L L L L L L L L L L L L L L L L L L L L L L L L L L L L L L L L L L L L L H H H H H H H H H H H H H H H H H H H H H H H H H H H H H H H H H H H H H H H H H H H H H H H H H H H H H H H H H H H H H H H H H H H H H H H H H H H H H H H H H H H H H H H H H H H H H H H H H H H H H H H H H H H H H H H H H H H H H H H H H H H H H H H H H H H H H H H H H H H H H H H H H H H H H H H H H H H H H H H H H H H H H H H H H H H H H H H H H H H H H H H H H H H H H H H H H H H H H H H H H L L L L L L L L L L L L L L L L L L L L L L L L L L L L L L L L L L L L L L L L L L L L L L L L L L L L L L L L L L L L L L L L L L L L L L L L L L L L L L L L L L L L L L L L L L L L L L L L L L L L L L L L L L L L L L L L L L L L L L L L L L L L L L L L L L L L L L L L L L L L L L L L L L L L L L L L L L L L L L L L L L L L L L L L L L L L L L L L L L L L L L L L L L L L L L L L L L L L L L L L L L L L L L L L L L L L L L L L L L L L L L L L L L L L L L L L L L L L L L L L L L L L L L L L L L L L L L L L L L L L L L L L L L L L L L L L L L L L L L L L L L L L L L L L L L L L L L L L L L L L L L L L L L L L L L L L L L L L L L L L L L L L L L L L L L L L L L L L L L L L L L L L L L L L L L L L L L L L L L L L L L L L L L L L L L L L L L L L L L L L L L L L L L L L L L L L L L L L L L L L L H H H H H H H H H H H H H H H H H H H H H H H H H H H H H H H H H H H H H H H H H H H H H H H H H H H H H H H H H H H H H H H H H H H H H H H H H H H H H H H H H H H H H H H H H H H H H H H H H H H H H H H H H H H H H H H H H H H H H H H H H H H H H H H H H H H H H H H H H H H H H H H H H H H H H H H H H H H H H H H H H H H H H H H H H H H H H H H H H H H H H H H H H H H H H H H H H H H H H H H H H L L L L L L L L L L L L L L L L L L L L L L L L L L L L L L L L L L L L L L L L L L L L L L L L L L L L L L L L L L L L L L L L L L L L L L L L L L L L L L L L L L L L L L L L L L L L L L L L L L L L L L L L L L L L L L L L L L L L L L L L L L L L L L L L L L L L L L L L L L L L L L L L L L L L L L L L L L L L L L L L L L L L L L L L L L L L L L L L L L L L L L L L L L L L L L L L L L L L L L L L L L L L L L L L L L L L L L L L L L L L L L L L L L L L L L L L L L L L L L L L L L L L L L L L L L L L L L L L L L L L L L L L L L L L L L L L L L L L L L L L L L L L L L L L L L L L L L L L L L L L L L L L L L L L L L L L L L L L L L L L L L L L L L L L L L L L L L L L L L L L L L L L L L L L L L L L L L L L L L L L L L L L L L L L L L L L L L L L L L L L L L L L L L L L L L L L L L L L L L L L L H H H H H H H H H H H H H H H H H H H H H H H H H H H H H H H H H H H H H H H H H H H H H H H H H H H H H H H H H H H H H H H H H H H H H H H H H H H H H H H H H H H H H H H H H H H H H H H H H H H H H H H H H H H H H H H H H H H H H H H H H H H H H H H H H H H H H H H H H H H H H H H H H H H H H H H H H H H H H H H H H H H H H H H H H H H H H H H H H H H H H H H H H H H H H H H H H H H H H H H H H L L L L L L L L L L L L L L L L L L L L L L L L L L L L L L L L L L L L L L L L L L L L L L L L L L L L L L L L L L L L L L L L L L L L L L L L L L L L L L L L L L L L L L L L L L L L L L L L L L L L L L L L L L L L L L L L L L L L L L L L L L L L L L L L L L L L L L L L L L L L L L L L L L L L L L L L L L L L L L L L L L L L L L L L L L L L L L L L L L L L L L L L L L L L L L L L L L L L L L L L L L L L L L L L L L L L L L L L L L L L L L L L L L L L L L L L L L L L L L L L L L L L L L L L L L L L L L L L L L L L L L L L L L L L L L L L L L L L L L L L L L L L L L L L L L L L L L L L L L L L L L L L L L L L L L L L L L L L L L L L L L L L L L L L L L L L L L L L L L L L L L L L L L L L L L L L L L L L L L L L L L L L L L L L L L L L L L L L L L L L L L L L L L L L L L L L L L L L L L L L L H H H H H H H H H H H H H H H H H H H H H H H H H H H H H H H H H H H H H H H H H H H H H H H H H H H H H H H H H H H H H H H H H H H H H H H H H H H H H H H H H H H H H H H H H H H H H H H H H H H H H H H H H H H H H H H H H H H H H H H H H H H H H H H H H H H H H H H H H H H H H H H H H H H H H H H H H H H H H H H H H H H H H H H H H H H H H H H H H H H H H H H H H H H H H H H H H H H H H H H H H L L L L L L L L L L L L L L L L L L L L L L L L L L L L L L L L L L L L L L L L L L L L L L L L L L L L L L L L L L L L L L L L L L L L L L L L L L L L L L L L L L L L L L L L L L L L L L L L L L L L L L L L L L L L L L L L L L L L L L L L L L L L L L L L L L L L L L L L L L L L L L L L L L L L L L L L L L L L L L L L L L L L L L L L L L L L L L L L L L L L L L L L L L L L L L L L L L L L L L L L L L L L L L L L L L L L L L L L L L L L L L L L L L L L L L L L L L L L L L L L L L L L L L L L L L L L L L L L L L L L L L L L L L L L L L L L L L L L L L L L L L L L L L L L L L L L L L L L L L L L L L L L L L L L L L L L L L L L L L L L L L L L L L L L L L L L L L L L L L L L L L L L L L L L L L L L L L L L L L L L L L L L L L L L L L L L L L L L L L L L L L L L L L L L L L L L L L L L L L L L L L H H H H H H H H H H H H H H H H H H H H H H H H H H H H H H H H H H H H H H H H H H H H H H H H H H H H H H H H H H H H H H H H H H H H H H H H H H H H H H H H H H H H H H H H H H H H H H H H H H H H H H H H H H H H H H H H H H H H H H H H H H H H H H H H H H H H H H H H H H H H H H H H H H H H H H H H H H H H H H H H H H H H H H H H H H H H H H H H H H H H H H H H H H H H H H H H H H H H H H H H H L L L L L L L L L L L L L L L L L L L L L L L L L L L L L L L L L L L L L L L L L L L L L L L L L L L L L L L L L L L L L L L L L L L L L L L L L L L L L L L L L L L L L L L L L L L L L L L L L L L L L L L L L L L L L L L L L L L L L L L L L L L L L L L L L L L L L L L L L L L L L L L L L L L L L L L L L L L L L L L L L L L L L L L L L L L L L L L L L L L L L L L L L L L L L L L L L L L L L L L L \\
%	data\_in[7:0] & 205D{0x0} 602D{0x1} 602D{0x2} 602D{0x3} 1004D{0x4} \\
%	state[5:0] & 3015U \\
%	re\_res[7:0] & 2413U 602D{0x5} \\
%	im\_res[7:0] & 2413U 602D{0x6} \\
%	LED[7:0] & 2215D{0x0} 198U 202D{0x5} 201D{0x6} 199D{0x0} \\
%	re\_a[7:0] & 208U 2807D{0x1} \\
%	im\_a[7:0] & 810U 2205D{0x2} \\
%	re\_q[7:0] & 1412U 1603D{0x3} \\
%	im\_q[7:0] & 2014U 1001D{0x4} \\
\end{tikztimingtable}

	\caption{\texttt{test\_sm.sv} Output}
	\label{fig:test-sm}
\end{figure}

%%%%%%%%%%%%%

\section{Synthesis}  \label{sec:synth}
\review{Include ``Synthesis Result''}


\section{Conclusion}  \label{sec:conclusion}

